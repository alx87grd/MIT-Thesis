\chapter{Experimental Validation}
\label{sec:ExperimentalValidation}

This chapter present a novel robot prototype using DSDM actuators, and experimental validations of the proposed control schemes.


\begin{figure}[htp]
	\centering
		\includegraphics[width=0.60\textwidth]{arm_proto_3.jpg}
	\caption{3-DoF custom arm using 3 DSDM actuators}
	\label{fig:dsdm_arm}
\end{figure}


\section{DSDM-arm}
\label{sec:DSDMArm}

A custom robotic arm using 3 DSDM actuators was designed and built, see Fig. \ref{fig:dsdm_arm}. This arm is very strong and fast for its weight when compared to traditional robotic arms. 

\subsection{Actuator design}
\label{sec:ActuatorDesign}

Three actuator prototypes were developed, with different mechanical advantage for the shoulder, elbow and wrist DoF of the arm. All of them use discrete components for the ease of implementation and modularity. The shoulder actuator is designed to drive a ballscrew, for a large efficient reduction, and the others actuators are embedded into revolute joints. 

\subsubsection{Revolute Joint Actuators}

Fig. \ref{fig:dsdm_act} shows the prototype for a revolute DSDM actuator. The elbow and wrist actuator have the same design with the exception of using different gearing ratios.

\begin{figure}[htp]
	\centering
		\includegraphics[width=0.75\textwidth]{dsdm_proto_2.jpg}
	\caption{Revolute joint DSDM actuator prototype } %Max continuous torque of 40 Nm and maximum velocity of 100 RPM
	\label{fig:dsdm_act}
\end{figure}

The prototype consist of a custom housing holding both the differential and support bearing for the output. Discrete \textit{Maxon} motors with gear-head of the serie GP32 can be attached to the back of the gear-box. It is thus possible to attach a wide-range of motor, from 20 watts to 200 watts, and with a wide range of additional gear-head reduction. 

The differential in the gear-box is implemented using a planetary gear-train, where is ring-gear is mounted on bearings. The planet-carrier unit is connected to the actuator output, the sun-gear is connected to the high-speed motor, and the ring-gear is driven by an additional reduction stage connected to the high-force motor. Fig. \ref{fig:dsdm_section} show the internal architecture of the system.


\begin{figure}[htp]
	\centering
		\includegraphics[width=0.95\textwidth]{metal_DSDM_section_view.jpg}
	\caption{Section view of the CAD model of the actuator prototype} %Max continuous torque of 40 Nm and maximum velocity of 100 RPM
	\label{fig:dsdm_section}
\end{figure}


\subsubsection{Linear Actuator}

To achieve the large reduction needed for the shoulder actuator of the robot, while keeping the mechanism back-drivable during high-speed mode, a large-lead ballscrew linear stage is used. 


\subsubsection{Specifications}


\subsection{Arm design}
\label{sec:ArmDesign}


%Shown in Fig. \ref{fig:arm_proto}, a 3-DoF robotic arm using variable gear-ratio actuators has been designed and custom built. The variable gear-ratio actuators consist of dual-speed dual-motor (DSDM) actuators, as shown in Fig. \ref{fig:dsdm_cad}. Those actuators have two discrete operating modes, one for high-speed operation, and the other for low-speed, high-load operation, where the net gear-ratios are more than 20-times different. DSDM actuators are lighter than would be single motors sized to reach the same maximum speed and maximum torque \cite{girard_two-speed_2015}. Also, the dual-motor architecture has the advantages of allowing fast and seamless gear-shifts, by doing the synchronization in the nullspace, which support the modeling assumption discussed at sec. \ref{sec:mod}. 


%For instance, the actuation and joint unit shown at Fig. \ref{fig:dsdm_cad} weights 1.5 Kg (un-optimized first prototype), can reach a 100 RPM velocity and a maximum continuous torque of 40 Nm (based on the very conservative torque rating of Maxon motor [ref] )

%\begin{figure}[htp]
	%\centering
		%\includegraphics[width=0.45\textwidth]{arm_proto_3.jpg}
	%\caption{Prototype 3-DOF arm with variable gear-ratio actuators}
	%\label{fig:arm_proto}
%\end{figure}



\section{Control architecture}
\label{sec:ControlSoftwareArchitecture}

The control algorithms are implemented on a computer using ROS \cite{quigley_ros:_2009}, the trajectory generation algorithm and the R* Computed Torque controller are written in \textit{Python}. The computer is communicating over USB with open-source \textit{Flexsea} motor-drivers \cite{duval_flexsea-execute:_2016} that handle the low-level current loops. The trajectory is generated offline in advance and loaded in memory upon initialization. The main R* control loop is running at a 500 Hz sampling rate. The low-level actuator controllers, receiving the torque and gear-ratio set-points, communicating with both motor drivers and handling the gear-shifting process (see Fig. \ref{fig:control_achitecture}), are ... TODO

%also implemented in \textit{Python} and use the algorithm described in \cite{girard_two-speed_2015}. Transition from one gear-ratio to another are found to be consistently under 50 ms.

\begin{figure}[htp]
	\centering
		\includegraphics[width=0.95\textwidth]{ros_diagram.png}
	\caption{Control Software Architecture}
	\label{fig:ros_diagram}
\end{figure}


\subsubsection{Motor drivers}

\subsubsection{Actuator Controllers}

\subsubsection{R* Robot Controller}

\subsubsection{Motion Planning Algorithm}


\section{Experimental results}
\label{sec:ExperimentalResults}




\subsection{Seamless gearshifts}
\label{sec:SeamlessGearshifts}

\subsection{Dynamic motions}
\label{sec:DynamicMotions}


Here a trajectory following experiments using the last DoF of the robot only is presented. A 1.5 Kg load is mounted on the end-effector, and the task is to bring it from the bottom position ($q=-\pi$) to the up-right position ($q=0$) using as little torques as possible. An RRT trajectory planning algorithm is used to search for a low torque trajectory reaching the goal, see Fig. \ref{fig:exp_rrt}. Then the R* Computed Torque Controller is used to track the reference trajectory. The experimental results are shown in Fig. \ref{fig:exp_traj} and a video of this experiment is also available in the multimedia attachment. 
%
\begin{figure}[htp]
	\centering
		\includegraphics[width=0.75\textwidth]{rrt_fig.pdf}
	\caption{Trajectory generation algorithm searching for a low torque solution}
	\label{fig:exp_rrt}
\end{figure}
%
\begin{figure}[htp]
	\centering
		\includegraphics[width=0.75\textwidth]{exp_fig3.pdf}
	\caption{Experimental results}
	\label{fig:exp_traj}
	\vspace{-10pt}
\end{figure}
%
Results show that the robot is using its 1:23 gear-ratio to accumulate kinetic energy by swinging the arm link back and forth. Also the R* controller selects the 1:474 gear-ratio automatically to attenuate the load dynamics, when the actuator has to force the robot to stay with the trajectory.  Interestingly, the reference trajectory was planned so the robot would accumulate enough kinetic energy to swing straight up with the last swing. However, in the experiment, the dissipative forces are greater than anticipated by the planner, and the last swing is too small (the robot almost stop at $q=-0.9$ at $t=2.6$ in Fig. \ref{fig:exp_traj}). Then, the R* controller automatically engage the large 1:474 gear-ratio, to continue converging on the desired trajectory with much smaller torques than those required if keeping using the 1:23 gear-ratio in this situation (no momentum and a large gravitational force to overpower). This illustrates that including the gear-ratio selection in the feedback loop also increase the robustness of the system. Without the 1:474 gear-ratio option, tracking would have failed as the computed torque with 1:23 in this situation was greater than the maximum allowable motor torque.


