\chapter{The DSDM-Lightweight Arm \\ { \large Mechanical Design, Control and Software Architecture } }
\label{sec:ExperimentalValidation}

{
\begin{flushright}
\textit{"What I cannot create, I do not understand."} \\ 
\emph{-- Richard Phillips Feynman}
\end{flushright}
}
\vspace{10pt}

%\begin{flushright}
%\small"The test of all knowledge is experiment. Experiment is the sole judge of scientific truth." \\ \emph{Richard Phillips Feynman}
%\end{flushright}


This chapter present a novel robot prototype using DSDM actuators, see Fig. \ref{fig:dsdm_arm_zoom} and Fig. \ref{fig:dsdm_arm}. The mechanical design of the DSDM actuators and the robotic arm is discussed, as well as the control and software implementation.

\begin{figure}[htb]
	\centering
		\includegraphics[width=0.70\textwidth]{arm_proto_zoom.jpg}
	\caption{one joint of the DSDM-Arm}
	\label{fig:dsdm_arm_zoom}
\end{figure}

\begin{figure}[htp]
	\centering
		\includegraphics[width=0.95\textwidth]{arm_proto_3.jpg}
	\caption{3-DoF custom arm using 3 DSDM actuators}
	\label{fig:dsdm_arm}
\end{figure}


\section{Mechanical Design}

\subsection{DSDM Actuator Design}
\label{sec:ActuatorDesign}
 
Three actuator prototypes were developed for the shoulder, elbow and wrist DoF of the arm, with different mechanical advantages. The shoulder actuator is designed to drive a ballscrew, for a large efficient reduction, and the others actuators are embedded into revolute joints. 

\subsubsection{3-port Differential gear-box}

On

\begin{figure}[htp]
        \centering
				\subfloat[Linear Actuator]{ % intrinsic dynamics
				\includegraphics[width=0.40\textwidth]{gears.jpg} }
				\hspace{+5pt}
        \subfloat[Revolute Joint]{ % intrinsic dynamics
				\includegraphics[width=0.40\textwidth]{gears_new.jpg} }
        \caption{Differential gear-box implemented with a planetary}
				\label{fig:differnentials}
\end{figure}


\subsubsection{Brake}


\subsubsection{Revolute Joint Actuators}

Fig. \ref{fig:dsdm_act} shows the prototype for a revolute DSDM actuator. The elbow and wrist actuator have the same design with the exception of using different gearing ratios.

\begin{figure}[htp]
	\centering
		\includegraphics[width=0.50\textwidth]{dsdm_proto_2.jpg}
	\caption{Revolute joint DSDM actuator prototype } %Max continuous torque of 40 Nm and maximum velocity of 100 RPM
	\label{fig:dsdm_act}
\end{figure}

The prototype consist of a custom housing holding both the differential and support bearing for the output. Discrete \textit{Maxon} motors with gear-head of the serie GP32 can be attached to the back of the gear-box. It is thus possible to attach a wide-range of motor, from 20 watts to 200 watts, and with a wide range of additional gear-head reduction. 

\begin{figure}[htbp]
	\centering
		\includegraphics[width=0.90\textwidth]{dsdm_parts.jpg}
	\caption{Dsdm parts}
	\label{fig:dsdm_parts}
\end{figure}


The differential in the gear-box is implemented using a planetary gear-train, where is ring-gear is mounted on bearings. The planet-carrier unit is connected to the actuator output, the sun-gear is connected to the high-speed motor, and the ring-gear is driven by an additional reduction stage connected to the high-force motor. Fig. \ref{fig:dsdm_section} show the internal architecture of the system.


\begin{figure}[htp]
	\centering
		\includegraphics[width=0.95\textwidth]{metal_DSDM_section_view.jpg}
	\caption{Section view of the CAD model of the actuator prototype} %Max continuous torque of 40 Nm and maximum velocity of 100 RPM
	\label{fig:dsdm_section}
\end{figure}


\subsubsection{Linear Actuator}

To achieve the large reduction needed for the shoulder actuator of the robot, while keeping the mechanism back-drivable during high-speed mode, a large-lead ballscrew linear stage is used. 

\begin{figure}[htp]
	\centering
		\includegraphics[width=0.95\textwidth]{proto_linear.pdf}
	\caption{linear actuator assembly} 
	\label{fig:linact}
\end{figure}

\begin{figure}[htbp]
	\centering
		\includegraphics[width=0.90\textwidth]{dsdm_parts_old.jpg}
	\caption{Dsdm parts}
	\label{fig:dsdm_parts_old}
\end{figure}


\subsubsection{Specifications}

For instance, the actuation and joint unit shown at Fig. \ref{fig:dsdm_cad} weights 1.5 Kg (un-optimized first prototype), can reach a 100 RPM velocity and a maximum continuous torque of 40 Nm (based on the very conservative torque rating of Maxon motor [ref] 


\subsection{Arm Design}
\label{sec:DSDMArm}

A custom robotic arm using 3 DSDM actuators was designed and built, see Fig. \ref{fig:dsdm_arm}. This arm is very strong and fast for its weight when compared to traditional robotic arms. 


\subsubsection{Shoulder 4-bar mechanism}

\begin{figure}[H]
	\centering
		\includegraphics[width=0.95\textwidth]{4bar.jpg}
	\caption{Shoulder 4-bar mechanism}
	\label{fig:4bar}
\end{figure}



\section{Control and Software Architecture}
\label{sec:ControlSoftwareArchitecture}

\subsection{Architecture}

\begin{figure}[H]
	\centering
		\includegraphics[width=0.95\textwidth]{ros_diagram.png}
	\caption{Control Software Architecture}
	\label{fig:ros_diagram}
\end{figure}


The control algorithms are implemented on a computer using ROS \cite{quigley_ros:_2009}, the trajectory generation algorithm and the R* Computed Torque controller are written in \textit{Python}. The computer is communicating over USB with open-source \textit{Flexsea} motor-drivers that handle the low-level current loops. The trajectory is generated offline in advance and loaded in memory upon initialization. The main R* control loop is running at a 500 Hz sampling rate. The low-level actuator controllers, receiving the torque and gear-ratio set-points, communicating with both motor drivers and handling the gear-shifting process (see Fig. \ref{fig:control_achitecture}), are ... TODO

%also implemented in \textit{Python} and use the algorithm described in \cite{girard_two-speed_2015}. Transition from one gear-ratio to another are found to be consistently under 50 ms.

\begin{figure}[H]
	\centering
		\includegraphics[width=0.95\textwidth]{ros_graph_3_dsdm.png}
	\caption{ROS architecture for controlling the full robot}
	\label{fig:ros_3dsdm}
\end{figure}


\begin{figure}[H]
	\centering
		\includegraphics[width=0.95\textwidth]{ros_graph_single_dsdm.png}
	\caption{ROS architecture for controlling a single DSDM actuator directly}
	\label{fig:ros_dsdm}
\end{figure}


\subsection{Motor drivers}

The first generation NI Labview

Scaling to 6 motors and ROS connectivity

Second gen use micro-controllers \cite{duval_flexsea-execute:_2016} 

\subsection{DSDM Actuator Controllers}



\subsection{Robot Controller}

\subsection{Trajectory Planning }

