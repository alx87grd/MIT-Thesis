\chapter{Optimal Dynamic Selection of Gear-ratios}  % to exploit or attenuate the external load dynamics
\label{sec:ControlAndPlanningOfRobotUsingVariableGearRatioActuators}

%\begin{flushright}
%\textit{"With four parameters I can fit an elephant, and with five I can make him wiggle his trunk."} \\ \emph{John von Neumann}
%\end{flushright}

\begin{flushright}
\textit{"He will win who knows when to fight and when not to fight."} \\ \emph{-- Sun Tzu}
\end{flushright}
\vspace{10pt}


The transmission gear-ratio that couples an actuator to a load has a significant effect upon the behavior of the actuator-load system. With a large reduction ratio, the load-side dynamics has no significant effect because it is attenuated by the factor of the square of the gear-ratio. The net load acting on the actuator is mostly its own intrinsic load, including rotor inertia and friction. In contrast, with a small reduction ratio or a direct drive system \cite{asada_direct-drive_1987}, the behavior is usually dominated by the load-side dynamics which consist of highly non-linear inertial and gravitational forces for robotics manipulators. Sometime it can be advantageous to exploit the load-side dynamics: gravity may push the robot in a desired direction; the robot may coast with small dissipative torques induced at the actuator side; or the robot joints become backdrivable to comply to an external force. In other situations, however, it may be advantageous to isolate the actuators from the load-side dynamics and external disturbances: using a large gear-ratio to bear a large load or moving it slowly against gravity, for example.

This chapter aims to explore the potentials of actuator transmissions that can be switched dynamically, such as the technology presented in chapter \ref{sec:MultipleSpeedActuationTechnology}, to either attenuate or leverage the natural dynamics of the system. Robots using lightweight VGA have the potential for achieving fast motions, high load-bearing, compliance and high-impedance, as required by diverse load conditions encountered by robotic systems. However, to truly exploit those salient features of VGA, control laws to select dynamically gear-ratios based on the current situation and task of the robot must be developed. Here in this chapter, feedback laws for robot control including gear-ratio selection are thus explored. The variable gear-ratio is used not merely for increasing maximum torque and speed, but also to significantly alter the dynamic properties, including load sensitivity, robustness, and backdrivability advantageously given the situation.

%Outline
In section \ref{sec:princ}, the principle of load leveraging and attenuation is delineated for a simple 1-DoF manipulator. Section \ref{sec:chal} will discuss related works and technical challenges. Section \ref{sec:model} will introduce a formal mathematical representation and propose a dynamic model for robotic system using VGA. Two different control approaches are explored, a model-based control synthesis in section \ref{sec:HierachicalControlApproach}, and a computational approach in section \ref{sec:DynamicProgrammingAproach}. The advantage of actively changing the gear-ratio are then illustrated with simulations in section \ref{sec:shift_sim}, and with experiments using a custom robotic arm in section \ref{sec:shift_exp}.

\subsection{Illustration of the principle for a 1-DoF manipulator}
\label{sec:princ}

Fig. \ref{fig:bigpicture} illustrates a simplified 1-DoF robotic manipulator where an electric motor is coupled to a pendulum through a gearbox with a gear-ratio $R$.

\begin{figure}[H]
 %\vspace{-10pt}
	\centering
		\includegraphics[width=0.75\textwidth]{gearratioeffect.pdf}
	\caption{Effect of the gear ratio on the dynamics}
	%\vspace{-10pt}
	\label{fig:bigpicture}
\end{figure}

As illustrated by phase portraits in Fig. \ref{fig:pp}, if $R$ is small then the dynamic behavior of the system is dominated by the non-linear pendulum dynamics (Fig. \ref{fig:pp1}), but if $R$ is very large, the behavior is dominated by the intrinsic inertia of the actuator, leading to the double-integrator behavior (Fig. \ref{fig:pp2}).

\begin{figure}[H]
				%\vspace{-10pt}
        \centering
				\subfloat[ Reduction ratio $R$=1 ]{ %extrinsic dynamics
				\includegraphics[width=0.40\textwidth]{pp1_hand.pdf}
				\label{fig:pp1}}
        \subfloat[Reduction ratio $R$=10 ]{ % intrinsic dynamics
				\includegraphics[width=0.40\textwidth]{pp10_hand.pdf}
				\label{fig:pp2}}
        \caption{Phase portraits illustrating the dynamical behavior}
				\label{fig:pp}
\end{figure}

The vector fields of Fig. \ref{fig:pp} illustrate the evolution of the system with no actuator torque. Suppose that we want to move from state A to state B on the phase plane. Starting off state A with the gear-ratio of 1:1 brings the system along the curved trajectory shown in Fig. \ref{fig:pp1}. Switching the gear ratio at state C to 1:10 will change the trajectory to the one in Fig. \ref{fig:pp2}, and bring the system to the destination state B. Note that no actuator torque is necessary for following this trajectory. A salient feature of actively changing the gear-ratio is that a vector field having properties useful for moving in a desired direction can be selected to minimize the necessary torque to apply to the system. 


\subsection{Challenges and related works}
\label{sec:chal}

This chapter investigates control scheme for robots equipped with variable gear-ratio actuators. This differs from variable stiffness actuators that use a variable transmission placed between a compliant element and the load \cite{vanderborght_variable_2013}, where it influence the reflected output stiffness but not the steady-state power transmission characteristics. While variable gear-ratio actuators have been studied extensively for automobile power-trains, they have not yet been fully investigated in robotics, despite significant potential gains. Variable gear-ratio transmissions for electric motors have been proposed for legged locomotion \cite{hirose_design_1991}, grasping robotic hands \cite{shin_robot_2012} , propulsion system \cite{lee_new_2012} \cite{mckeegan_antonovs_2011} and actuation systems \cite{girard_two-speed_2015} \cite{hirose_development_1999} \cite{tahara_high-backdrivable_2011}. Some of those works address the issue of how to change the gear-ratio, but there is no general approach to the high-level control of automatically selecting the right gear-ratio for nonlinear, coupled multi-DoF robotic systems.


From the control perspective, automating the gear-ratio selection in a robotic context is a new and challenging problem. Gear-shifting is a very non-linear process and the plant becomes a hybrid dynamical system if the usable gear-ratios are a set of discrete values. Hence most control engineering tools are not suited to tackle this problem. 

In simple scenarios, the gear-ratio selection can be based on simple principles. For instance, for a system running at a steady speed, the best gear-ratio can be selected based on efficiency. Alternatively, for rapid acceleration, the gear-ratio may be selected based on the actuator-load inertia matching \cite{giberti_effects_2010} \cite{chen_generalized_1991}. A multi-DoF robot, however, experiences diverse types of forces acting simultaneously. These include gravity, friction, and inertial forces as well as Coriolis and centrifugal forces. Hence, it is challenging to find a general control policy for selecting gear-ratios for the multitude of dynamically interacting actuators in the robotics context. 

Mixed-integer programming has been used to generate optimal open-loop trajectories of complex dynamical system with both continuous torque and discrete gear-selection input variables \cite{gerdts_solving_2005}. However, open-loop trajectories can be unstable and new trajectories must be computed for each initial goal and target pair. 


Dynamic programming was used to generate feedback laws for torque and gear-ratio selection in simple systems \cite{girard_practical_2016}. However, this technique is computationally expensive, it does not scale well to high-dimensional systems, and it required a significant amount of offline computation for each different goal position. 


Here in this paper, a model-based approach is proposed, with the advantage of scaling to high-dimensional robotics system. The method is applied to trajectory tracking, and the entire trajectory and gear-ratio control law is synthesized so that small actuators can dynamically lift a heavy load. 

The focus will be on systems where the gear-ratio options are limited to a discrete set, but the proposed principle and control algorithms are also applicable to robots with continuously variable transmissions.

TODO incorporate this:

Discrete mode of operations are present in many system, such as powertrains with multiple gear ratios, pneumatic or hydraulic systems equipped with on/off valves and special actuators \cite{leach_linear_2012}\cite{Wolfgang_novel_2015}\cite{lee_finger_2013}. Considering the global behavior of such systems, including the mode selection, lead to a hybrid dynamical model. Most optimal control techniques are based on either variational approaches or some form of gradient descent to find a trajectory that minimize a cost function \cite{betts_practical_2010}. Hence those techniques cannot be used directly to optimize discrete variables. An interesting approach to get around this problem is to use the switching instants as optimization parameters instead \cite{xu_optimal_2004}\cite{majdoub_hybrid_2010}. However, to use this approach a sequence of operating modes must be predefined first. Mixed-integer programming has also been used to generate optimal open-loop trajectories of dynamical system with both continuous and discrete input variables \cite{richards_spacecraft_2002} \cite{gerdts_solving_2005}. However, open-loop trajectories can be unstable and new trajectories must be computed for each initial condition. Hence, from a practical point of view, techniques generating feedback policies are preferable. Most of the analytical results regarding feedback control of hybrid systems are for specific cases, for instance the optimal feedback law of linear hybrid systems with linear constraints for a quadratic cost function have been shown to have a particular form \cite{borrelli_dynamic_2005}. One computational technique that generate feedback laws and that can be used for non-linear systems with any kind of constraints is dynamic programming \cite{donald_e._kirk_optimal_2004}. Two disadvantages of the techniques are however that it only works for low-dimensional systems (so called curse of dimensionality) and also that the resulting feedback laws are in the form of a look-up table.


\newpage

\section{Control architecture}
\label{sec:arch}

\begin{figure}[H]
				%\vspace{-10pt}
        \centering
				\subfloat[ Model-based approach ]{ %extrinsic dynamics
				\includegraphics[width=0.45\textwidth]{archi3.png}
				\label{fig:archi3}}
				\hspace{+10pt}
        \subfloat[ Dynamic programming approach ]{ % intrinsic dynamics
				\includegraphics[width=0.45\textwidth]{archi2.png}
				\label{fig:archi2}}
        \caption{Proposed Control Architectures}
				\label{fig:controlarchitectures}
\end{figure}

TODO incorporate this:

The proposed control architecture, shown in Fig. XXX, consists of three hierarchical control loops. First, a trajectory generation algorithm synthesizing dynamic trajectories that meets performance requirements for reaching desired states. Second, a closed-loop trajectory following controller consisting of a feedback law computing actuator torques $\vec{\tau}$ and gear-ratios $R$ based on the measured full state of the robot. Finally at the lowest level, independent actuator controllers executing low-level hardware commands in response to given torques and gear-ratio set-points. 

The main scope of this paper is the trajectory following controller: the design of a feedback law optimizing gear-ratios and computing torque commands in real-time. For the trajectory generation, motion planning algorithms can be applied to optimize the overall trajectory \cite{lavalle_planning_2006}.  The low-level controllers would be specific to the type of actuator used in the system. It is assumed that they would track the desired torque/force and handle the gear-shifting process when receiving a new gear-ratio set-point. 


\newpage

\section{Modeling Variable Gear-Ratio Actuators}
\label{sec:model}

\begin{flushright}
{%\footnotesize %\small
\textit{"With four parameters I can fit an elephant, and with five \\ I can make him wiggle his trunk."}
 }
 \emph{-- John von Neumann}
\end{flushright}
\vspace{+10pt}

In this section, a simple approach is proposed for modeling robot using variable gear-ratio actuators. Variable transmission are modeled as variable transformer elements, using the bond-graph terminology. This representation allows for a clear physical understanding of the effect of gear-ratio even for non-linear $n$-DoF systems. Furthermore, this modeling approach facilitates the implementation of a real-time optimization in the proposed controller. Limitations are discussed at section \ref{sec:limitation}.

%\begin{table}[htbp]
	%\centering
		%\begin{tabular}{ c c l }
		%
        %\hline \hline
			%$H$             &  :  & External inertia matrix \\
			%$D$             &  :  & External damping matrix \\
			%$C$             &  :  & External Coriolis/Centrifugal forces matrix  \\
			%$\vec{f}_g$     &  :  & External gravitational forces vector  \\
			%$\vec{d  }$     &  :  & External disturbances forces vector  \\
			%$R$             &  :  & Gear-ratio matrix (diagonal) \\
			%$I$             &  :  & Intrinsic actuator inertia matrix (diagonal) \\
			%$B$             &  :  & Intrinsic actuator damping matrix (diagonal) \\
			%$\vec{\tau}$    &  :  & Electromagnetic motor torques  \\
			%$\vec{q}$       &  :  & Joint coordinates position vector  \\
			%$\vec{w}$       &  :  & Actuator coordinates velocity vector  \\
			%$J$             &  :  & task-space coordinates / joint coordinates jacobian matrix\\
			%%&\text{Secondary variables:  \\
			%$\vec{f}$       &  :  & Net transmitted forces (joint coordinates) \\
			%$\vec{\tau}'$   &  :  & Net transmitted forces (actuators coordinates) \\
			%$\vec{\tau}_I$  &  :  & Sum of intrinsic forces  \\
			%$\vec{\tau}_E$  &  :  & Sum of extrinsic forces  \\
		%\hline \hline
        %\end{tabular}		
        %\caption{Nomenclature}	% Table caption must be placed on top of the table %
	%\label{tab:nom}
%\end{table}

\subsection{1-DoF system}
\label{sec:1DOFSystem}

First, a generic 1-DoF robot with a variable transmission is considered for simplicity. If the actuator's intrinsic resistive forces $\tau_I$ are approximated to a linear quantity, the equations of motion (EoM) can be written as:
%
\begin{align}
  H \ddot{q} + D \dot{q} + g( q )	&= R  \left[ \tau - I \dot{w} - B w	\right] \\
	\underbrace{\left[	H \ddot{q} + D \dot{q} + g( q )	\right]}_{\tau_{E}(\ddot{q},\dot{q},q)}
	&= R \tau - R^2
	\underbrace{\left[ I \ddot{q} + B \dot{q}	\right]}_{\tau_{I}(\ddot{q},\dot{q})} \\
	\tau &= 	\frac{\tau_{E}(\ddot{q},\dot{q},q)}{R} + R \; \tau_{I}(\ddot{q},\dot{q})
	\label{eq:1dofEoM}
\end{align}
%
where the effect of the gear ratio can be seen clearly; increasing $R$ attenuates the external dynamic terms $\tau_{E}$ but amplify the intrinsic actuator losses $\tau_{I}$ for a given trajectory.
%
Variable gear-ratios can be modeled as variable transformer elements, using the bond-graph terminology. Fig. \ref{fig:bondgraph1} shows a bond-graph representation of the model.
\begin{figure}[htp]
	\centering
		\includegraphics[width=0.70\textwidth]{bondgraph1.pdf}
	\caption{Model of a 1-DoF robot with a variable gear-ratio actuator}
	\label{fig:bondgraph1}
\end{figure}


\subsection{Generalization to $\boldsymbol{n}$-DoF manipulators}
\label{sec:GeneralizationToNDOFManipulators}

To generalize the above model to a $n$-DoF system with $n$ actuators, the load-side dynamics is considered as a generic form of manipulator equations where each port is connected to an independent actuator through a network of transformers. The network of transformers can be view as a type of coordinate transformation relating effort (force or torque) and flow (velocity or angular velocity) on the load side ($\vec{f}$,$\dot{\vec{q}}$) to those on the actuator output side ($\vec{\tau}'$,$\vec{w}$):
%
\begin{align}
	\vec{ f } = R^T \vec{\tau}' \quad  \quad R \dot{ \vec{q} } = \vec{w}
 \label{eq:coortransform}
\end{align}
%
where $R$ is a $n$ by $n$ matrix consisting of all the transformer ratios. 
%
\begin{figure}[htp]
	\centering
		\includegraphics[width=0.80\textwidth]{bondgraph.pdf}
	%\vspace{-10pt}
	\caption{Model of a $n$-Dof robot with variable actuator-joint coupling}
	\label{fig:bondgraph}
\end{figure}

The EoM are then given by:
%
\begin{align}
	&\underbrace{ H \vec{ \ddot{q} } + C\vec{ \dot{q} } + D \vec{ \dot{q} } + \vec{ g } }_{ \vec{\tau}_{E}(\ddot{\vec{q}},\dot{\vec{q}},\vec{q})}
		= R^T \underbrace{  \left[ 
		\vec{ \tau } - I \vec{ \dot{w} } - B \vec{ w }       
		\right]}_{ \vec{\tau}' } 
 \label{eq:eom_ndof}
\end{align}
%
Note that, in the case of locomotion or manipulation where the robot interacts with the environment by physically contacting it, the dynamic model $\vec{\tau}_{E}$ must reflect the contact conditions, either by computing contact forces using constraint equations or by formulating $\vec{\tau}_{E}$ as a hybrid dynamical system.

In most practical cases, each actuator has its independent variable transmission and, thereby, the $R$ matrix will be diagonal and each diagonal value can be selected independently. Assuming this situation, the EoM can be simplified to a form, similar to the scalar case, illustrating the effect of the gear ratios matrix $R$: 
%
\begin{align}
	\vec{\tau} &= R^{-1} 
	\underbrace{ 
	\vec{\tau}_{E}(\ddot{\vec{q}},\dot{\vec{q}},\vec{q}) 
	}_{\text{External load dynamics}}
	+ R 
	\underbrace{ 
	\vec{\tau}_{I}(\ddot{\vec{q}},\dot{\vec{q}})
		}_{\text{Intrinsic losses}}
	\\ %&\text{where} \;
	\vec{\tau}_{I} &\triangleq I \vec{ \ddot{q} } + B \vec{ \dot{q} } 
	%[ \vec{\tau}_{I} ]_j &\triangleq [I]_{jj} [\vec{ \ddot{q} }]_j + [B]_{jj}  [ \vec{ \dot{q} }]_j  \quad \forall j \in {1,2,...,n}
 \label{eq:eom_ndof2}
\end{align}
%
The derivation of this simplified form is available in the Appendix \ref{sec:Rdiagndof}.

\subsection{Limitation of the simplified model}
\label{sec:limitation}
%
The main assumption of the proposed model is that gear-ratios are considered as independent control inputs, neglecting all the dynamics and delays associated with the transition of gear-ratios. Physically this implies that the kinetic energy of the system may be discontinuous at a gear-shift since the energy necessary for the transition is not considered. In the case of a car transmission for instance, this model would not keep track of the energy used for accelerating or braking the engine during the synchronization process. This model can be used if the gear-shift process is fast compared to the dynamics of the robots and if the energetic losses due to the gear-shift are negligibly small. In addition this model also assumes that all motor rotors are in an inertial reference frame, neglecting gyroscopic effects, which may be induced when the axes of motor rotors are rotated.

\subsection{Uncertainty}
\label{sec:uncertainty}

Two observations are made regarding the effect of the gear-ratios on disturbances. First, considering modeling errors and external forces on the extrinsic side as unknown generalized forces $\vec{d}$, the EoM given by eq. \eqref{eq:eom_ndof2} becomes:
\begin{align}
	\vec{\tau} &= R^{-1} 
	\vec{\tau}_{E}(\ddot{\vec{q}},\dot{\vec{q}},\vec{q}) 
	+ R 
	\vec{\tau}_{I}(\ddot{\vec{q}},\dot{\vec{q}})
    + R^{-1}
    \underbrace{ 
	\vec{d}
	}_{\text{Disturbances}}    
 \label{eq:eom_ndof3}
\end{align}
where it is assumed that the actuators are accurately modeled. Note that the effect of the disturbances is inversely proportional to the gear-ratios, and thereby attenuated when using large gear-ratios. 

Second, large gear-ratios also decrease the sensitivity of the system to uncertainty. The error of computed accelerations will be attenuated with large gear-ratios because of the larger  actuator inertia reflected to the extrinsic side:
\begin{align}
	\vec{\ddot{q}}_e &= \vec{\ddot{q}} - \vec{\ddot{q}}_r = 
	\left[ 
    H + R^T I_a R
	\right]^{-1}
    \vec{d}
 \label{eq:sens}
\end{align}
%
Hence, selecting large gear-ratios makes the system less sensitive to uncertainty on the extrinsic side.


\newpage

\section{Model-based control approach}
\label{sec:HierachicalControlApproach}


\begin{flushright}
\textit{"If you know the enemy and know yourself, you need \\ not fear the result of a hundred battles."}  \emph{-- Sun Tzu}
\end{flushright}
\vspace{+10pt}

In this section, control algorithms relying on a dynamic model of a robot and its load are proposed. Here, a methodology is presented to synthesize feedback laws, for both the torques and gear-ratios input variable, to follow a trajectory with minimal effort. 

\subsection{Optimal Gear ratio along a trajectory}

This section analyzes the optimal gear-ratio at each instant along a known trajectory. 

\subsubsection{Selection criteria}
\label{sec:GearSelectionCriteria}

The two main advantages of changing gear-ratio are 1) lowering the necessary torque to follow a trajectory and 2) modifying the effective impedance reflected on the environment. 

\paragraph{Torque}
Optimization for reducing torque can be done by minimizing $\vec{\tau}^T \vec{\tau}$ at each point along the trajectory. 

\paragraph{Impedance}
Optimization for reflected impedance can be done by minimizing the difference between desired task-space impedance and the actual one, which is directly affected by the matrix $R$. For instance, the end-point inertia matrix contains the gear ratios: 
%
\begin{align}
	M = [J(\vec{q})^T]^{-1} \big [ \underbrace{ R^T I R }_{\text{Actuator contribution}} + H( \vec{q} ) \big ] J(\vec{q})^{-1}
 \label{eq:endpointmass}
\end{align}
%
Another point of practical importance is that the rotor speed should be constrained to be lower than their maximum velocity. This is to avoid infeasible gear shifts. For example, attempting to shift to a low gear at an extremely high speed is impossible. 

\subsubsection{Optimization}
 The optimal gear-ratio is determined by minimizing the total actuator torques and, optionally, the difference in end-point impedance:
%
\begin{align}
	R^{*}(\ddot{\vec{q}},\dot{\vec{q}},\vec{q}) &= \operatornamewithlimits{argmin}\limits_{R} \left[ \vec{\tau}^T \vec{\tau} + \alpha \| M_{d} - M \| \right]  \\
	& \text{s.t}  \quad R \dot{\vec{q}} \leq \vec{w}_{max} 
\label{eq:rmin_general}
\end{align}
%
where $\alpha$ is a parameter to set the trade-off between minimizing motor torques and matching the desired end-point inertia.
%
For a 1-DoF system, the optimal gear ratio leading to minimal torque, not considering any velocity constraints, at a given instant on a trajectory is given by
%
\begin{align}
	R^{*} &= \operatornamewithlimits{argmin}\limits_{R} \left[ \tau^2 \right] = \sqrt{ \left | \frac{\tau_{E}(\ddot{q},\dot{q},q)}{\tau_{I}(\ddot{q},\dot{q})} \right |   } 
\label{eq:aaa}
\end{align}
%
The derivation is available in the Appendix \ref{sec:optgearproof1}.

Similarly for a multi-DoF system, if $R$ is a diagonal matrix, the optimal gear-ratios can be obtained independently for each axis:
%
\begin{align}
	%R^{*} = \operatornamewithlimits{argmin}\limits_{R} \left[  \vec{\tau}^T \vec{\tau} \right] \Rightarrow 
	[R^*]_{ii} = \sqrt{ \left | \frac{ [\vec{\tau}_{E}(\ddot{\vec{q}},\dot{\vec{q}},\vec{q})]_i }{ [\vec{\tau}_{I}(\ddot{\vec{q}},\dot{\vec{q}})]_i } \right | }
 \label{eq:rmin2}
\end{align}
%
The derivation is available in the Appendix \ref{sec:optgearproofn}.

Note that large gravitational forces or external disturbances, only present in $\vec{\tau}_{E}$, will usually lead to larger optimal gear-ratios, unless they cancel-out other forces in a way that makes $\vec{\tau}_{E}$ smaller. If inertial or viscous forces, present both in $\vec{\tau}_{E}$ and $\vec{\tau}_{I}$, dominate, then the optimal gear-ratio will be a compromise such that extrinsic and intrinsic forces are balanced, a form of impedance matching. The optimal gear ratio given by \eqref{eq:rmin2} includes both gravity, inertial and viscous effects as well as all other effects, hence it can be applied to any arbitrary dynamic situations.

\newpage



\subsection{Trajectory following controllers}

\subsubsection{R* Computed Torque}
\label{sec:RobustTrajectoryFollowingController}


The proposed closed-loop controller, shown in Fig. \ref{fig:Rstar_block_big}, is based on the Computed Torque technique \cite{asada_robot_1986}, but includes an optimization step to compute and select the optimal gear-ratios. The idea is as follow, first compute a necessary acceleration $\ddot{\vec{q}}_r$ to guarantee convergence on the trajectory. Then given the actual position $\vec{q}$, actual speed $\dot{\vec{q}}$ and desired acceleration $\ddot{\vec{q}}_r$ compute the optimal gear-ratios $R^*$, as described previously for a known trajectory, and apply the corresponding necessary torques $\vec{\tau}^*$. As illustrated, the model-based estimation of extrinsic and intrinsic forces can optionally be improved by using a disturbance observer. The salient feature of the R* controller is that the optimal gear-ratio is selected based on state-feedback, i.e. even in situations not foreseen in the planner that generated the nominal trajectory. For instance, if a disturbance pushes the robot in a state where the robot faces a large gravitational force requiring a large gear-ratio, the controller will automatically select it. Similarly if facing a contact forces, if it is included in the model or estimated with a disturbance observer, the R* controller will automatically select the appropriate gear-ratio. Fig. \pageref{fig:dq} offer a graphical interpretation of the R* algorithm in the phase plane. 

\begin{figure}[t]
	\centering
		\includegraphics[width=0.99\textwidth]{Rstar_block_big.pdf}
	\caption{R* Computed Torque Controller}
	%\vspace{-10pts}
	\label{fig:Rstar_block_big}
\end{figure}

\begin{figure}[htp]
	\centering
		\includegraphics[width=0.40\textwidth]{dq.png}
	\caption[R* algorithm graphical interpretation]{The R* algorithm can be interpreted graphically, as selecting the gear-ratio for which the natural acceleration vector is the closet to the desired acceleration vector $\vec{\ddot{q}}_r$ (after scaling the distance with the inertia), in order to minimize the necessary torques to apply on the system.}
	\label{fig:dq}
\end{figure}

\subsubsection{Robust Control}
\label{sec:robustcontrol}

In general Computed Torque Control is susceptible to modeling uncertainties and disturbances. This section presents two approaches to improving robustness: Adaptation and Sliding Mode Control. 

\paragraph{Adaptation}
If the uncertainty is structured as unknown model parameters in the extrinsic dynamics, the term represented by $\vec{\tau}_E$, then traditional adaptation schemes can be used for estimating the unknown parameters. Then, if adaptation converge to the correct computed torque, then the computed best gear-ratios will also converge to the true optimal gear-ratios:
\begin{align}
	\hat{\vec{\tau}}_E \rightarrow \vec{\tau}_E 
    \quad \Rightarrow \quad 
    \hat{R}^* \rightarrow R^*
 \label{eq:adapt}
\end{align}

\paragraph{Sliding Mode Control}

Alternatively, if the uncertainty (disturbances, noise and modeling errors) is bounded, Sliding Mode Control can be applied to improving robustness. %the gear-ratio selection can also be used to minimize the torques required to guarantee convergence to the trajectory. 
Introducing intermediary variables defined as:
\begin{align}
	\vec{q}_e        = \vec{q} - \vec{q}_d  \quad \quad
	\vec{s}          = \dot{\vec{q}}_e + \lambda \vec{q}_e \quad \quad
  \vec{\ddot{q}}_r = \ddot{\vec{q}}_d - \dot{\vec{q}}_e
 \label{eq:slidingvar}
\end{align}

the computed torque, eq. \eqref{eq:eom_ndof2}, is modified to the following sliding mode control law: 
\begin{align}
	\vec{\tau} &=  R^{-1} 
	\vec{\tau}_{E}(\ddot{\vec{q}}_r,\dot{\vec{q}},\vec{q}) 
	+ R 
	\vec{\tau}_{I}(\ddot{\vec{q}}_r,\dot{\vec{q}})
    + R^{-1} G sgn( \vec{s} ) 
 \label{eq:slidingctl}
\end{align}
Convergence to the desired trajectory, despite the uncertainty, can be guaranteed using the Lyapunov function $V=\vec{s}^T \vec{s}$. The sliding condition is guarantee \cite{asada_robot_1986}, for any selected gear-ratios, if the discontinuous gain is set to:
\begin{align}
	G &= \left[ H + R^T I_a R \right] diag( \vec{k} ) \\ k_{i} &> \max \left| \left(  \left[ H + R^T I_a R \right]^{-1} \vec{d} \right)_{i} \right|
 \label{eq:slidingcond}
\end{align}
%
If the gear-ratio is selected to minimize the sliding mode torque, eq. \eqref{eq:slidingctl}, instead of the computed torque, eq. \eqref{eq:eom_ndof2}, then naturally larger gear-ratio  will be selected in response to large uncertainty. For instance, for a 1-DoF case:
\begin{align}
	R^{*} &= \operatornamewithlimits{argmin}\limits_{R} \left[ \tau^2 \right] = \sqrt{ \left | \frac{\tau_{E}(\ddot{q}_r,\dot{q},q) + | d |_{max} \, sgn( s ) }{\tau_{I}(\ddot{q}_r,\dot{q})} \right |   } 
\label{eq:rstar_sliding}
\end{align}
Hence, if no disturbance is expected ($| d |_{max}=0$) gear-ratio selection is unaffected, but when large disturbances are expected ( $| d |_{max}$ is large) torque minimization naturally leads to selecting large gear-ratios.
%



\subsubsection{Numerical Optimization}

If some of the assumptions used in the previous section are not valid or if the gear-ratios are limited to a few discrete choices, then the optimization must be computed numerically. The minimum needed is to have a model of the inverse dynamic, which could include any non-linearity, in the form:
\begin{align}
	\vec{\tau}  = f( \vec{\ddot{q}} , \vec{\dot{q}} , \vec{q} , \vec{d} , R ) 
\end{align}
In the situation of discrete gear-ratios, this lead to a combinatorial optimization problems, adding a constraint on the gear selection in the form of:
\begin{align}
	%R^{*} &= \operatornamewithlimits{argmin}\limits_{R} \left[  \vec{\tau}^T  \vec{\tau} + \alpha \| M_{d} - M \|  \right] \quad \\
	&  R \in \{R_1,R_2, ... , R_l\} 
\end{align}
However, if the number of options $l$ is reasonably small, then every possible options can be computed quickly. For instance, for the robot presented in this paper, see Fig. \ref{fig:arm_proto}, there is 3 actuators each with 2 gear-ratio options, leading to $l=2^3=8$ possible matrix $R$.



\subsubsection{Heuristic approach to minimize gearshift}

Because the control effort during gear-shifts is neglected in the model, using the proposed controller can lead to rapid switching between gear-ratios in certain situations. To avoid this undesirable behavior, it is proposed to add hysteresis to the controller: the gear-ratio is only changed if the difference of computed torque, between using the optimal and the previously selected gear-ratio, is greater than a minimum torque threshold, and also if the elapsed time since the last change is greater than a minimum delay. 



\subsubsection{Examples}
\label{sec:Examples}

Here eq. \eqref{eq:aaa} is applied to the robot in Fig. \ref{fig:bigpicture} in simple scenarios. 

\paragraph{Acceleration from rest} 

When the robot accelerates from rest with no viscous forces, the optimal gear ratio at the up-right position, where no gravity acts, is given by:
\begin{align}
	R^{*}  = \sqrt{ \left | \frac{H \ddot{q} }{ I \ddot{q} } \right |   } = \sqrt{ \frac{H}{I}}
 \label{eq:impmatching}
\end{align}
In this situation, the problem is reduced to impedance matching for two inertial loads. The optimal gear ratio minimizing the torque for a given acceleration is the one for which the load inertia and the motor reflected inertia are the same.

\paragraph{Supporting gravity without moving}

In the situation where the robot is not moving and fighting against gravity, then the optimal gear ratio is:
\begin{align}
	R^{*}  = \sqrt{ \left | \frac{ f_g }{ 0 } \right |   } \rightarrow \infty
 \label{eq:gravrejection}
\end{align}
In this static case, the largest possible gear-ratio is the optimal choice. 

\paragraph{Resisting disturbances}
%
In the situation where there is no gravity and the robot is not moving, but disturbances are expected, minimizing the sliding mode torque (eq. \eqref{eq:rstar_sliding}) would also lead to the conclusion that the largest possible gear-ratio is the optimal choice.
\begin{align}
	R^{*}  = \sqrt{ \left | \frac{ 0 + | d |_{max} \, sgn( s ) }{ 0 } \right |   }  = \sqrt{  \frac{| d |_{max} }{ 0 } }\rightarrow \infty
 \label{eq:gravrejection}
\end{align}



\subsection{Trajectory planning}
\label{sec:SamplingBasedTrajectoryPlanner}

\newpage

\section{Dynamic programming approach}
\label{sec:DynamicProgrammingAproach}

\begin{flushright}
\small"The true logic of this world is in the calculus of probabilities." \\ \emph{-- James Clerk Maxwell}
\end{flushright}

This paper address the design of global feedback laws, for both continuous and discrete inputs, that lead to a desired global behavior, which will be formulated as reaching a target while minimizing a cost function. A practical computational approach that put simplicity before absolute optimality is proposed. The technique could be used on any low-dimensional system for a wide range of control objectives. Here, this paper focus on demonstrating the approach for the task of driving a 11.4 kg (25 lbs) mass for point-to-point motions using the two-speed actuator shown at Fig. \ref{fig:proto_linear}. 

The proposed approach is to discretize the continuous control problem into a graph search problem where the discrete input actions can be considered naturally, and then use dynamic programming to find global optimal feedback policies. First, a simplified low-dimensional model of the controlled system is derived to keep the problem tractable. Second, the optimal control problem is solved using dynamic programming. Third, the resulting optimal policies are approximated with simple feedback laws, to conduct a stability analysis and also for the ease of implementation.

The control problem of obtaining the global desired behavior is formulated as minimizing a scalar cost $J$ that is a function of the state trajectory $\underline{x}(:)$ and the inputs trajectory $\underline{u}(:)$, while constraining both states and inputs to be in their respective domains:
\begin{align}
	\operatornamewithlimits{min}\limits_{\underline{u}(:)} \; & J(\underline{x}(:),\underline{u}(:)) = \int g(\underline{x}(t),\underline{u}(t)) dt  \\
	s.t. \quad & \underline{\dot{x}} = F(\underline{x}, \underline{u}) \\
	& u_1 \in [ -0.02 , 0.02 ] \quad (Nm) \\
	& u_2 \in \{ 1 , 2 \} \\
	& x_1 \in [ -150 , 150 ] \quad (mm) \\
	& x_2 \in [ -500 , 500 ] \quad (mm/sec)
	\label{eq:min}
\end{align}

An additional constraint is imposed, when the output speed is greater than 20 mm/sec only the operating mode $u_1=1$ can be used (to limit motor speeds to 8000 RPM). Then three different additive cost functions are investigated; a quadratic cost function $g_q$ where error is weighted against control effort, a function corresponding $g_t$ to the minimal time problem and a function $g_e$ corresponding to thermal losses in the motors, leading to a simplified minimum wasted energy problem (neglecting mechanical losses).
%
\begin{align}
	g_q(\underline{x},\underline{u}) &= w_1 \; x^2 + w_2 \; \dot{x}^2 + w_3 \; u_1^2
	\label{eq:g_quad} \\
	g_t(\underline{x},\underline{u}) &= 1
	\label{eq:g_time} \\
	g_e(\underline{x},\underline{u}) &= u_1^2 \quad  \propto \quad  \dot{Q}_{motor} = r I^2
	\label{eq:g_e}
\end{align}
%

\begin{figure}[H]
	\centering
		\includegraphics[width=0.45\textwidth]{blocks2.pdf}
	\caption{Controller architecture}
	\label{fig:blocks}
\end{figure}


\subsection{Value Iteration}
\label{sec:VI}

A dynamic programming algorithm, also known as value iteration, is used to find both the optimal cost-to-go function and optimal policies for an infinite horizon. The optimal control problem is numerically solved for the different three cost functions. The discretization parameters are as follow: the time step is 0.02 sec, the state space is discretized into an even 501 x 501 grid and the continuous torque is discretized into 51 discrete control options, for a total of 102 possible control actions including the mode selection. Artificial out-of-bound and on-target termination states are included to guarantee convergence \cite{DP}.

\subsection{Reinforcement Learning}

TODO get from class project


\subsection{Numerical results}

\subsubsection{Linear 1-DoF robot}
Fig. \ref{fig:J}-\ref{fig:u1} illustrate the numerical results of the dynamic programming algorithm for the three different cost functions. The absence of color indicates states with no solution (a constraint will be violated for any possible control actions). Fig. \ref{fig:phase_plane} shows the closed loop behavior of the system in the phase plane when the optimal policy is applied.

\begin{figure}[H]
        \centering
				\subfloat[Minimum time]{
        \includegraphics[width=0.32\textwidth]{Jt.png}
				\label{fig:J_time}}
        \subfloat[Quadratic cost]{
				\includegraphics[width=0.32\textwidth]{Jq.png}%J_LQR.png
				\label{fig:J_LQR}}
				\subfloat[Minimum energy]{
				\includegraphics[width=0.32\textwidth]{Je.png}
				\label{fig:J_energy}}
        \caption{Optimal cost-to-go $J^*$}\label{fig:J}
\end{figure}

\begin{figure}[H]
        \centering
				\subfloat[Minimum time]{
        \includegraphics[width=0.32\textwidth]{u1t.png}
				\label{fig:u0_time}}
        \subfloat[Quadratic cost]{
				\includegraphics[width=0.32\textwidth]{u1q.png}
				\label{fig:u0_LQR}}
				\subfloat[Minimum energy]{
				\includegraphics[width=0.32\textwidth]{u1e.png}
				\label{fig:u0_energy}}
        \caption{Optimal policy for the continuous torque command $u_1$ [Nm]}\label{fig:u0}
\end{figure}

\begin{figure}[H]
        \centering
				\subfloat[Minimum time]{
        \includegraphics[width=0.32\textwidth]{u2t.png}
				\label{fig:u1_time}}
        \subfloat[Quadratic cost]{
				\includegraphics[width=0.32\textwidth]{u2q.png}
				\label{fig:u1_LQR}}
				\subfloat[Minimum energy]{
				\includegraphics[width=0.32\textwidth]{u2e.png}
				\label{fig:u1_energy}}
        \caption{Optimal policy for the mode selection $u_2$}\label{fig:u1}
\end{figure}

\begin{figure}[H]
        \centering
				\subfloat[Minimum time]{
        \includegraphics[width=0.32\textwidth]{ppt.pdf}
				\label{fig:phase_plane_time}}
        \subfloat[Quadratic cost]{
				\includegraphics[width=0.32\textwidth]{ppq.pdf}
				\label{fig:phase_plane_LQR}}
				\subfloat[Minimum energy]{
				\includegraphics[width=0.32\textwidth]{ppe.pdf}
				\label{fig:phase_plane_energy}}
        \caption[Closed loop behavior in the phase plane]{Closed loop behavior with the optimal policy illustrated in the phase plane}\label{fig:phase_plane}
\end{figure}

\paragraph{Minimum time}
\label{sec:MinimumTime}
For the minimum time problem, the optimal policy is a bang-bang law for $u_1$ and always using highly-geared mode when possible. Note that the bang-bang switching curve accounts for the fact the large gear ratio will be used during the final part of the trajectory.

\paragraph{Quadratic cost}
\label{sec:QuadraticCost}
For the quadratic cost, the mode selection optimal policy is almost as simple as the minimum time problem except for small features in quadrant II and IV. The more interesting result comes from the continuous torque control law, the gains when using the large reduction ratio are larger than those when using the small reduction ratio. This results in the controller taking action mainly at low speed when its actions have the biggest impacts on the system, and lead to a highly non-linear closed-loop behavior. This is the opposite of what would have been obtained using feedback linearization: large gains with the small gear and small gains with the large gear to linearize the global behavior. Hence implementing feedback linearization would have led to a poor performance considering this quadratic cost metric. 

\paragraph{Minimum energy}
\label{sec:MinimumEnergy}
For the minimum energy controller, interestingly the mode selection policy is not trivial even for this simple linear model.  This shows that it does not take much complexity to have non-trivial optimal policies for hybrid systems. Here, the large reduction ratio is used almost only for braking, and in quadrant II and IV the small reduction ratio is used even at low speed to coast with low viscous resistance. Also globally the gains are much lower than the other controllers except for zones where it is necessary to use energy to stay in the domain. 

\subsubsection{Non-linear 1-DoF robot}

TODO Insert VI results for pendulum.

\subsection{Feedback laws simplification}

TODO Using regresstion to obtain analytical feedback laws...

In this section, simplified algebraic control laws are extracted from the numerical map of optimal control actions for the quadratic cost controller. The dynamic programming algorithm outputs a list of optimal actions that is assumed are samples of a  map of optimal actions $\underline{u}^* = \pi^* ( \underline{x} )$. The goal is then to find a simplified map $\hat{\pi}( \underline{x} )$ based on the samples. Here a two-steps approach is used; first the state-space is segmented into zones based on the optimal discrete action $u_2$, then regression is used to compute the $u_1$ map independently in each zone. 

\subsubsection{1-DoF linear robot example}

\paragraph{Segmentation}
For the resulting optimal policy found at Fig. \ref{fig:u0_LQR}, three main zones can be identified. The boundaries are approximated by two lines at $\dot{x}= \pm \; 20 $, and the samples are separated into two groups:
\begin{align}
\left \{ \underline{u}^i , \underline{x}^i \right \} \rightarrow \text{Group 1} \quad \text{if} \; | \dot{x} | \geq 20 \\
\left \{ \underline{u}^i , \underline{x}^i \right \} \rightarrow \text{Group 2} \quad \text{if} \; | \dot{x} | < 20
\end{align}

For more complex segmentation, machine learning techniques could be used to perform the classification.

\paragraph{Continuous maps}

In each zones, the optimal continuous input $u_1^*$ is very close to a planar surface with exceptions when the input saturates or near the boundaries, see Fig. \ref{fig:u0_LQR}. Here, it is proposed to forgo absolute optimality and instead use the simplest control law that can globally approximate $u_1^*$ map. Hence, simple linear plane are fitted on the samples group by group. The regression is defined as follow, the dependent variable to approximate is $u_1$, the independent variables are the state coordinates $\underline{x}^T = [ x  \; \dot{x} ]$ and the parameter vector is $\underline{\alpha}^T = [ k_p  \; k_d ]$. Hence the continuous control variable $\hat{u}_1$ is approximated with the following linear map:
\begin{align}
\hat{u}_1 = \underline{\alpha}^T \underline{x} = k_p  x + k_d \dot{x}
\end{align}
which is essentially a simple proportional-derivative control law. Then the parameters are estimated using a least-square criterion, $\underline{\alpha}_1$ using the samples in group 1 and $\underline{\alpha}_2$ samples in group 2. Then combining the segmentation and both linear maps, the global control policy is approximated with the following law:
\begin{align}
\underline{ \hat{u}} &= \hat{\pi} ( \underline{x} )
 = \left \{ 
	\begin{array}{c}
	\left[
	\begin{array}{c}
		 \underline{\alpha}_1^T \underline{x} \\
		 1 
	\end{array} 
	\right] 		
	\text{if} \; | \dot{x} | \geq 20 \\ \\
		 \left[
		\begin{array}{c}
		 \underline{\alpha}_2^T \underline{x} \\ 
		 2
	\end{array}
		 \right]
		\text{if} \; | \dot{x} | < 20
	\end{array}
	\right.
	\label{eq:uhat}
\end{align}
Fig. \ref{fig:u0_LQR_approx} shows the resulting continuous control policy $\hat{u}_1$ for the full state-space, and is an approximation of Fig. \ref{fig:u0_LQR} map.

%\begin{figure}[H]
	%\centering
		%\includegraphics[width=0.30\textwidth]{u1aq.png}
	%\caption{Continuous map approximation for $u_1$}
	%\label{fig:u0_LQR_approx}
%\end{figure}

\begin{figure}[htpb]
				\vspace{-10pt}
        \centering
				\subfloat[Map for $u_1$ (Nm)]{
        \includegraphics[width=0.45\textwidth]{u1aq.pdf}
				\label{fig:u0_LQR_approx}}
        \subfloat[Phase plane behavior]{
				\includegraphics[width=0.40\textwidth]{pp2q.pdf}
				\label{fig:phase_plot_lqr}}
        \caption{Approximated control laws and resulting behavior}\label{fig:approx}
\end{figure}


The resulting behavior with the control law of eq.\eqref{eq:uhat} is illustrated in the phase plane at Fig. \ref{fig:phase_plot_lqr}, where the arrows illustrate the state derivative throughout the state space. The blue arrows illustrate the natural dynamic of the system and the red arrows illustrate the closed behavior with the control policy of eq.\eqref{eq:uhat}. At high speed with the small reduction ratio, the controller only takes small actions illustrated by the fact that red arrows only have small deviation compared to blue arrows. However at low speed, the large reduction ratio is used to drastically change the natural behavior of the system, illustrated by the large red arrows. 



%%%%%%%%%%%%%%%%%%%%%%%%%%%%%%%%%%%%%%%%%%%%%%%%%%%%%%%%%%%%%%%%%%%%%%%%%%%%%%%%%%%%%%%%%%%%%%%%%%%%%%%%%%%%%%%%%%%%%%%%%%%%%%%%%
\section{Simulation Results}
\label{sec:shift_sim}



\subsection{Model-based controller}


In this section, the advantages of dynamically changing the gear-ratios, using the R* computed torque controller, are illustrated using simulations of two robots: first a 1-DoF pendulum, then a 3-DoF arm. Both robots are considered having VGA with two possible gear-ratios: 1:1 or 1:10. Reference low-torque trajectories to reach target positions are computed offline using a sample-based search algorithm \cite{lavalle_planning_2006}. 
%
The first simulated experiment uses the robot of Fig. \ref{fig:bigpicture}, but here considering dissipative forces in the actuators, with the task of reaching the up-right position starting at the bottom. Fig. \ref{fig:sim1} shows the robot tracking the reference low-torque trajectory, where at first the robot accumulates energy, using the 1:1 gear-ratio, and then finishes the motion using the 1:10 gear-ratio. When the gravitational forces are pushing advantageous toward the trajectory the controller select the 1:1 gear-ratio, but when it is advantageous to fight the intrinsic actuator dynamics instead, the 1:10 gear-ratio is selected. 

%
\begin{figure}[htp]
	\centering
		\includegraphics[width=0.75\textwidth]{sim1.pdf}
	\caption{1-DoF robot simulation: states and inputs trajectory}
	\label{fig:sim1}
\end{figure}
%
%
%\begin{figure}[htp]
				%\vspace{-10pt}
        %\centering
				%\subfloat[ Reduction ratio $R$=1 ]{ %extrinsic dynamics
				%\includegraphics[width=0.23\textwidth]{simpp1.pdf}
				%\label{fig:pp1s}}
        %\subfloat[Reduction ratio $R$=10 ]{ % intrinsic dynamics
				%\includegraphics[width=0.23\textwidth]{simpp2.pdf}
				%\label{fig:pp2s}}
        %\caption{Trajectory superposed with natural dynamics vectors}
				%\label{fig:pps}
%\end{figure}
%
%\begin{figure}[htp]
	%\centering
		%\includegraphics[width=0.38\textwidth]{simpp1.pdf}
	%\caption{Trajectory superposed with natural dynamics vectors with R=1}
	%\label{fig:simpp1}
%\end{figure}
%
%\begin{figure}[htp]
	%\centering
		%\includegraphics[width=0.38\textwidth]{simpp2.pdf}
	%\caption{Trajectory superposed with natural dynamics vectors with R=10}
	%\label{fig:simpp2}
%\end{figure}
%
In the second experiment, a 3-DoF manipulator is tasked with going from configuration A to configuration B with the 3D trajectory shown at Fig. \ref{fig:3d_traj}. For this robot the controller is actively selecting the best gear-ratios matrix $R$ out of the possible $2^3=8$ options. Fig. \ref{fig:3d_u} shows the control inputs activity. During the initial falling-down phase, at around $t=1$, the robot is using 1:1 gear-ratios for all actuators, leveraging gravitational torques. In contrast, during the final lifting phase, at around $t=6$, the robot is using 1:10 gear-ratios for all actuators. 
%
%
%
\begin{figure}[htp]
	\centering
		%\includegraphics[width=0.45\textwidth]{3dtraj.jpg}
		\includegraphics[width=0.75\textwidth]{3d_traj.pdf}
	\caption{ 3-DoF robot simulation: 3D trajectory }
	\label{fig:3d_traj}
\end{figure}
%
%\begin{figure}[htp]
	%\centering
		%\includegraphics[width=0.40\textwidth]{u_no4.pdf}
	%\caption{ 3-DoF robot simulation: control inputs trajectory where \textit{Mode} represent the index of the selected gear-ratio matrix }
	%\label{fig:3d_u}
%\end{figure}
%
\begin{figure}[htp]
	\centering
		\includegraphics[width=0.75\textwidth]{3d_u.pdf}
	\caption{ 3-DoF robot simulation: control inputs trajectory}
	\label{fig:3d_u}
\end{figure}
%%
%




\subsection{Value iteration}




\subsection{Comparison}


To evaluate the performance gain of actively changing the gear-ratio, simulations with fixed gear-ratios are conducted where a regular computed torque controller tracks the same trajectories. Results are summarized in TABLE \ref{tab:MaximumTorqueComparison}, in terms of maximum absolute torque, which relates to the required size and weight of motors, and integral of torque squared, which relates to power consumption. Active gear-ratio selection is found to greatly improve both metrics, especially for the 3-DoF robot trajectory where the arm must both achieve high-speeds and also sustain a constant gravitational load at the final configuration. Note that in those simulations high-velocity with 1:10 reductions is inhibited by friction in the motors, no maximum rotor velocity is enforced. For the 3-DoF trajectory, active gear-shifting is found to reduce the maximum torque required by a factor two and the integral of the torque square by a factor 10, compared to any of the fixed-gear options. Those results show the potential of using variable gear-ratio transmissions for huge improvements in terms of actuator size and power consumption. Moreover, here in the simulations, the load was always the same manipulator in different dynamic situations. As discussed in section \ref{sec:app}, if the load dynamics is radically changing because of different contact situations with the environment, the performance gain of changing the gear-ratio could be even greater. 
%
\begin{table}[htp]
	\centering
		\begin{tabular}{ c c c c }
		\hline
		     & Fixed gear & Fixed gear & Active gearshifting \\
			& 1:1 &  1:10 &  1:1 or 1:10 \\
		\hline \hline
		\multicolumn{4}{c}{ Max Absolute Torque [Nm] } \\
		\hline \hline
		1-link robot  & 15 & 88 & 12 \\	
		3-link robot  & 24 & 42 & 12 \\	
		\hline \hline
		\multicolumn{4}{c}{ Torque squared integral $\int{ ( \vec{\tau}^T \vec{\tau} ) dt }$ } \\
		\hline \hline
		1-link robot  & 377  & 8133 & 226  \\	
		3-link robot  & 2774 & 3617 & 295  \\	
		\hline \hline
		\end{tabular}
	\caption{Required torque comparison}
	\label{tab:MaximumTorqueComparison}
	\vspace{-20pt}
\end{table}
%



%%%%%%%%%%%%%%%%%%%%%%%%%%%%%%%%%%%%%%%%%%%%%%%%%%%%%%%%%%%%%%%%%%%%%%%%%%%%%%%%%%%%%%%%%%%%%%%%%%%%%%%%%%%%%%%%%%%%%%%%%%%%%%%%%


\section{Experiments Results}
\label{sec:shift_exp}

\subsection{Model-based controller}

First a trajectory following experiments using the last DoF of the robot only is presented. A 1.5 Kg load is mounted on the end-effector, and the task is to bring it from the bottom position ($q=-\pi$) to the up-right position ($q=0$) using as little torques as possible. An RRT trajectory planning algorithm is used to search for a low torque trajectory reaching the goal, see Fig. \ref{fig:exp_rrt}. Then the R* Computed Torque Controller is used to track the reference trajectory. The experimental results are shown in Fig. \ref{fig:exp_traj} and a video of this experiment is also available in the multimedia attachment. 
%
\begin{figure}[htp]
	\centering
		\includegraphics[width=0.75\textwidth]{rrt_fig.pdf}
	\caption{Trajectory generation algorithm searching for a low torque solution}
	\label{fig:exp_rrt}
\end{figure}
%
\begin{figure}[htp]
	\centering
		\includegraphics[width=0.75\textwidth]{exp_fig3.pdf}
	\caption{Experimental results}
	\label{fig:exp_traj}
	\vspace{-10pt}
\end{figure}
%
Results show that the robot is using its 1:23 gear-ratio to accumulate kinetic energy by swinging the arm link back and forth. Also the R* controller selects the 1:474 gear-ratio automatically to attenuate the load dynamics, when the actuator has to force the robot to stay with the trajectory.  Interestingly, the reference trajectory was planned so the robot would accumulate enough kinetic energy to swing straight up with the last swing. However, in the experiment, the dissipative forces are greater than anticipated by the planner, and the last swing is too small (the robot almost stop at $q=-0.9$ at $t=2.6$ in Fig. \ref{fig:exp_traj}). Then, the R* controller automatically engage the large 1:474 gear-ratio, to continue converging on the desired trajectory with much smaller torques than those required if keeping using the 1:23 gear-ratio in this situation (no momentum and a large gravitational force to overpower). This illustrates that including the gear-ratio selection in the feedback loop also increase the robustness of the system. Without the 1:474 gear-ratio option, tracking would have failed as the computed torque with 1:23 in this situation was greater than the maximum allowable motor torque.

Fig. \ref{fig:rob} shows four additional experiments for demonstrating how disturbance rejection can be improved by using the sliding mode version of the R* controller. Here the controller is only given a simple fixed point-target in all cases. First, when a low uncertainty bound is given to the controller, the robot can reach its target when unloaded (a) but failed when an unknown (to the controller) 0.4 Kg load is added to the end-effector (b). However, when a larger uncertainty bound is given, the robot can reach its target in both cases, unloaded (c) and loaded (d). Note that the discontinuous torque required to guarantee convergence despite disturbances is greatly reduced when down-shifting to a large gear-ratio at low speeds. For an improved performance, smoothing techniques should be implemented to avoid exciting the unmodeled high-frequency modes. Videos of all the experiments discussed above, and additional demonstrations are available in the media attachment.

\begin{figure}[htp]
				%\vspace{-10pt}
        \centering
				\hspace{-10pt}
				\subfloat[]{ % intrinsic dynamics
				\includegraphics[width=0.30\textwidth]{fig16a.pdf} }
				\hspace{-5pt}
        \subfloat[]{ % intrinsic dynamics
				\includegraphics[width=0.20\textwidth]{fig16b.pdf} }
				\hspace{-5pt}
				\subfloat[]{ % intrinsic dynamics
				\includegraphics[width=0.20\textwidth]{fig16c.pdf} }
				\hspace{-5pt}
				\subfloat[]{ % intrinsic dynamics
				\includegraphics[width=0.20\textwidth]{fig16d.pdf} }
        \caption{Experiments with sliding mode version of the R* controller }
				\label{fig:rob}
\end{figure}
