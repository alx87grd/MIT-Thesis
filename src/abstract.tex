In many applications, robots have to bear large loads while moving slowly and also have to move quickly through the air with almost no loads. Those conflicting requirements often leads to the use of oversized and inefficient actuators, which is inhibitory particularly for mobile robots. Multiple gear ratios, like in a powertrain, address this issue by making possible to leverage or attenuate the natural dynamic of the load and by allowing an effective use of power over a wide range of output speed. This thesis aims at developing the technological solutions needed to use variable gear-ratio actuators in a robotic context. First, by addressing the issue of how to make fast and seamless gearshifts under diverse load conditions, with a solution based on a dual-motor actuator architecture and a control scheme using the nullspace. Second, by developing control algorithms that actively selection the gear-ratios to move with minimal motor torques and to adjust the output impedance appropriately given a task. Experiments using a novel lightweight 3-DoF manipulator using 3 custom-built dual-speed dual-motor actuators are presented. Finally, the advantages of the approach are illustrated with a legged locomotion case-study, where a leg must move quickly in the air, manage an impact with the ground and then bear a large load.
