\chapter{Vision for Aircraft Manufacturing Automation}
\label{sec:VisionForAircraftManufacturingAutomation}

%%% Citation
\begin{flushright}
{
%\textit{"The problems of the world cannot possibly be solved by skeptics or cynics \\ whose horizons are limited by the obvious realities. We need men who can  \\ dream of things that never were and ask why not?"}
\textit{"We need men who can dream of things that never were and ask why not?"}
  -- \emph{John F. Kennedy}
%
%\emph{John F. Kennedy}
}
\end{flushright}
\vspace{+10pt}
%%%%

This chapter illustrates robotic concepts to automate the manufacturing of large objects, such as airplanes, ships, buildings, etc. For those types of manufacturing tasks, unlike car production for instance where the part can be on an assembly line and surrounded by robots, robots must be able to go inside the large manufactured object. %Additionally, with typically slower rates of production and larger numbers of distinct required operations, 


\section{Current situation}
\label{sec:CurrentSituation}

Here, the current manufacturing approach and the challenge for automating the manufacturing and assembly of aircraft is briefly explored.

In the aerospace industry, one technical challenge in aircraft manufacturing automation is bringing robots inside the fuselage . Standard industrial robots are too large and heavy to be used efficiently inside the fuselage. 

Fig. \ref{fig:spyder} illustrate two possible solutions: one is a long, snake-like articulated arm \cite{buckingham_r._chitrakaran_v._conkie_r._ferguson_g._et_al._snake-arm_2007} and the other is a mobile robotic platform. In both cases the robot must bear the weight of its own actuation system, which increases exponentially along the serial linkage as the number of DoF increases. The use of variable gear-ratio actuator systems would solve or alleviate the actuator problem. 

%\cite{parietti_supernumerary_2014}
%\cite{menon_design_2011}

Aircraft manufacturing is largely dependent on manual labor due to the complexity of tasks, stringent inspection requirements, difficulties in installing a conveyor line, and small lot size. In a typical
aircraft assembly factory, a number of scaffolds are used to




\section{Solution Concepts}

Here, three class of possible solution are presented.




\subsection{Mobile climbing robots}
\label{sec:MobileClimbingRobots}

Another approach, aiming at a higher level of automation, is to have mobile robots walking or climbing inside the aircraft fuselage to reach manufacturing sites automatically. 


\begin{figure}[H]
	\centering
		\includegraphics[width=0.9\textwidth]{spyder_concept.png}
		\caption{Mobile climbing manufacturing robot concept}
	\label{fig:arm_concept}
\end{figure}


\subsection{Wearable robots}
\label{sec:WearableRobots}

One possible solution, to solve bring robot on site easily is to use the help of humans, which unlike robot would have no problems navigating and moving inside a manufacturing site.

\begin{figure}[H]
	\centering
		\includegraphics[width=0.6\textwidth]{wearables_robots.png}
		\caption{Wearable robot concepts and prototypes}
	\label{fig:wearable_concept}
\end{figure}


\subsection{Light-weight long manipulator arms}
\label{sec:LightWeightLongManipulatorArm}

Alternatively, robotic arm that would be very long and articulated could be able to reach manufacturing sites from the outside of the fuselage. 

\begin{figure}[H]
	\centering
		\includegraphics[width=1.00\textwidth]{arm_concept.png}
		\caption{Long light-weight arm concept for interior access}
	\label{fig:arm_concept}
\end{figure}