\chapter{Conclusion}
\label{sec:Conclusion}



%{
%\begin{flushright}
%\small
%\textit{"Scientists discover the world that exists; engineers create the world that never was."} \\ 
%\emph{Theodore Von Kármán}
%\end{flushright}
%}
%\vspace{+10pt}

%\begin{flushright}
%\small"For a successful technology, reality must take precedence over public relations, for nature cannot be fooled." \\ \emph{Richard Phillips Feynman}
%\end{flushright}

%“If a conclusion is not poetically balanced, it cannot be scientifically true.” 
%― Isaac Asimov

%“The easiest way to solve a problem is to deny it exists.” 
%― Isaac Asimov


This thesis explored the idea of robotic systems using actuators with variable transmissions, i.e. where the reduction ratio can be dynamically changed online. Although variable transmissions are used extensively in vehicle powertrains, this concept is highly under-explored and under-exploited in the field of robotics, despite huge potential gains as demonstrated in this thesis. Variable gear-ratios actuators can be used not merely for increasing maximum torque and speed, but also to significantly advantageously alter the dynamic properties of robots including load sensitivity, robustness and backdrivability. This thesis main contributions are 1) DSDM actuators: a solution to make gear-shifting transitions adapted to a wide range of robotic tasks, and 2) Control approaches to synthesized optimal closed-loop gear-ratios selection policies, for a very generic class of robotic systems using variable transmissions.

%
Chapter \ref{sec:VisionForAircraftManufacturingAutomation} briefly discussed manufacturing applications, where actuators must meet challenging requirements, which could hugely benefit from the proposed technologies developed in this thesis. 
%
Chapter \ref{sec:MultipleSpeedActuationTechnology} presented the DSDM actuation technology that can change its effective reduction ratio, between a small reduction and a very large reduction, quickly and seamlessly even in highly dynamic situations. 
%
Chapter \ref{sec:ControlAndPlanningOfRobotUsingVariableGearRatioActuators} explored the idea of closed-loop selection of gear-ratios for multi-DoF robotic systems, and proposed control schemes that leverage all the advantages offered by variable transmissions. Analytical optimal solutions, for a class of robotic systems, and guarantees in terms of stability and chattering behavior are also derived.
%
Chapter \ref{sec:ExperimentalValidation} presented the DSDM-Arm, a novel lightweight robotic system using three DSDM actuators, which was used for experimental validation of all the proposed control schemes and also to demonstrate the advantages of robotic systems equipped with variable gear-ratio actuators.
%
Multiple experiments with the DSDM actuators demonstrated the salient features and the ability of the DSDM technology to change gear ratio quickly and seamlessly even in very dynamic situations, including impacts. Simulations and experiments with the DSDM-Arm were presented and demonstrated that actively changing gear ratios using the proposed control algorithms can lead to an order-of-magnitude reduction of necessary motor torque and power.

The author would like encourage all researchers in the field of robotics, to question the single-gear electric-motor actuation paradigm, and envisioning variable transmissions for applications that require speed and force in a small package. The field is appealing from both an engineering and a scientific perspective. On one hand, it is a very practical solution to relevant power transmission problems: cars, bicycles, electric drills, etc., use multiple speed transmissions. On the other hand, it makes the problem of controlling non-linear robotic systems: even more non-linear and hybrid. Exciting research questions are raised that are both challenging and worth solving, and the author hopes you joint him in visiting this realm where there be dragons.






