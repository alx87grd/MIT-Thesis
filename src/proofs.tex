\chapter{Proofs}
\label{sec:proofs}


\section{Simplified equation of motion for diagonal $R$}
\label{sec:Rdiagndof}

Here the derivation of the simplified form of equations of motion, where intrinsic and extrinsic forces are two separate terms, is derived for arbitrary $n$-DoF fully actuated robots.

\paragraph{Assumptions} 

Two assumptions are necessary for this form. First, that a coordinates system can be selected so that the gear-ratio matrix $R$ is diagonal for any selected operating mode:
%
\begin{align}
R_{i,j} = 0 \quad \forall \; i \neq j
\end{align}
%

Second, that dynamic forces related to viscous damping and inertial forces from the rotor side are linear with respect to rotor velocity:
%
\begin{align}
\vec{\tau}_{rotor-inertia} = I \ddot{\vec{q}}  \quad  \vec{\tau}_{rotor-damping} = B \dot{\vec{q}}
\end{align}
%

Note that additionally, in all cases, matrix $I$ and $D$ are diagonal since motor rotors are not coupled directly. 

\paragraph{Derivation}

Starting from the general, eq. XX, the EoM are:
%
\begin{align}
H \vec{ \ddot{q} } + C\vec{ \dot{q} } + D \vec{ \dot{q} } + \vec{ g }
		= R^T  \left[ 
		\vec{ \tau } - I \vec{ \dot{w} } - B \vec{ w }       
		\right]
\end{align}
%
Then substituting motor velocity to joint coordinates using the kinematic relation, eq. XX,
%
\begin{align}
H \vec{ \ddot{q} } + C\vec{ \dot{q} } + D \vec{ \dot{q} } + \vec{ g }
		&= R^T  \left[ 
		\vec{ \tau } - I R \vec{ \ddot{q} } - B R\vec{ \dot{q} }       
		\right] \\
H \vec{ \ddot{q} } + C\vec{ \dot{q} } + D \vec{ \dot{q} } + \vec{ g }
		&= R^T \vec{ \tau } - R^T I R \vec{ \ddot{q} } - R^T B R \vec{ \dot{q} }  
\end{align}
%
Then because $R$, $I$ and $B$ matrix are diagonal, they can be permuted and also $R^T=R$. Hence, the EoM can be rearranged:
%
\begin{align}
H \vec{ \ddot{q} } + C\vec{ \dot{q} } + D \vec{ \dot{q} } + \vec{ g }
		&= R \vec{ \tau } - R R I \vec{ \ddot{q} } - R R B \vec{ \dot{q} }  
\end{align}
%
Then, assuming the robotic system is fully actuated, the R matrix is square and invertible. Then multiplying by $R^{-1}$ from the left on both side:
%
\begin{align}
R^{-1} \left[ H \vec{ \ddot{q} } + C\vec{ \dot{q} } + D \vec{ \dot{q} } + \vec{ g } \right]
		&= \vec{ \tau } - R I \vec{ \ddot{q} } - R B \vec{ \dot{q} }  
\end{align}
%
Then rearranging:
%
\begin{align}
R^{-1} \left[ H \vec{ \ddot{q} } + C\vec{ \dot{q} } + D \vec{ \dot{q} } + \vec{ g } \right]
		&= \vec{ \tau } - R  \left[ I \vec{ \ddot{q} } + B \vec{ \dot{q} }  \right]
\end{align}
%
and thus obtain the desired final form:
%
\begin{align}
\vec{ \tau } &=  R^{-1} \underbrace{ \left[ H \vec{ \ddot{q} } + C\vec{ \dot{q} } + D \vec{ \dot{q} } + \vec{ g } \right] }_{\vec{\tau}_E}
 + R \underbrace{ \left[ I \vec{ \ddot{q} } + B \vec{ \dot{q} }  \right]}_{\vec{\tau}_I}
\end{align}
%



\newpage
\section{Optimal gear-ratio along a known trajectory}
\label{sec:optgearproof}

Starting from the EoM, eq. XXX, assuming that the robot is fully actuated and viscous forces linear in speed, it is possible to derive closed form expression for the optimal gear-ratio.


\subsection{Single DoF}
\label{sec:optgearproof1}

Starting with the EoM in inverse dynamic form (from eq. XXX):
%
\begin{align}
\tau  &=  \frac{\tau_E}{R} + R \tau_I
\end{align}
%
Using a quadratic cost function to minimize:
%
\begin{align}
J &=  \tau^2 = \frac{\tau_E^2}{R^2} + 2 \tau_E \tau_I + R^2 \tau_I^2
\end{align}
%

Finding the gear-ratio that minimize this cost can be formulated as:
%
\begin{align}
R^* &=  \operatornamewithlimits{argmin}\limits_{R} J \\
   &   \text{s.t.} \quad R \in \Re  \quad \& \quad R > 0
\end{align}
%
A non-real value would have no physical sense. A negative $R$ value would be physically possible, for instance the reverse gear in a car. However, for symmetric electric motors, in the sense that they behave the same way for any sign of torque and speed, there should be no gain obtained by changing the direction of the motor velocity. This can be seen as the cost function is symetric with respect to $R$:
%
\begin{align}
J(R) = J( -R )
\end{align}
%

\paragraph{Derivation}

First finding the partial derivative of the cost $J$ with respect to $R$:
%
\begin{align}
\frac{ \partial J }{ \partial R  } &=  2 \tau \frac{ \partial \tau }{ \partial R } = 2 \left( \frac{\tau_E}{R} + R \tau_I \right) \left( -\frac{\tau_E}{R^2} + \tau_I \right) \\
\frac{ \partial J }{ \partial R  } &= 2 \left( R \tau_I^2 - \frac{\tau_E^2}{R^3} \right)
\end{align}
%
Then the second derivative:
%
\begin{align}
\frac{ \partial^2 J }{ \partial R^2  } &= 2 \left( \tau_I^2 + 3 \frac{\tau_E^2}{R^4} \right)
\end{align}
%
Hence, on the domain of interest, the second derivative is always positive:
%
\begin{align}
\frac{ \partial^2 J }{ \partial R^2  } >= 0  \quad \forall \quad R \in (0,+\infty)
\end{align}
%
Thus the cost function $J$ is convex in the desired interval of possible $R$ values. The minimum of the function can thus be found by solving for the point where the first derivative is equal to zero:
%
\begin{align}
0 &= \frac{ \partial J }{ \partial R  } = 2 \left( R \tau_I^2 - \frac{\tau_E^2}{R^3} \right) \\
R \tau_I^2 &= \frac{\tau_E^2}{R^3} 
\end{align}
%
Since $R>0$, it is possible to multiply both side by $R$, leading to
%
\begin{align}
R^4 \tau_I^2 &= \tau_E^2
\end{align}
%
Then, assuming a non-degenerative case of $\tau_I \neq 0$, it leads to
%
\begin{align}
R^4  &= \frac{\tau_E^2}{\tau_I^2} \\
R^2  &= \pm \sqrt{ \frac{\tau_E^2}{\tau_I^2} } = \pm \frac{\tau_E}{\tau_I} \\
R    &= \pm \sqrt{ \pm \frac{\tau_E}{\tau_I} } 
\end{align}
%
Then, the only real and positive solution to this equation is given by:
%
\begin{align}
R    &= \sqrt{ \left| \frac{\tau_E}{\tau_I} \right|} 
\end{align}
%

\paragraph{Solution}

The minimal cost value is thus obtain with the optimal gear-ratio value:
%
\begin{align}
R^*    &= \sqrt{ \left| \frac{\tau_E}{\tau_I} \right|} 
\end{align}
%
Which lead to the minimum cost:
%
\begin{align}
J^*    &=  2 \tau_E \tau_I  + 2 \left| \tau_E \tau_I \right| 
\end{align}
%
Note that the the minimized cost is zero when extrinsic and intrinsic forces have opposite signs.

\paragraph{Degenerative cases}
If the intrinsic forces are equal to zero, then the cost tends towards zero as the gear-ratio $R$ tends toward $\infty$:
%
\begin{align}
\tau_I = 0 \quad \& \quad R \rightarrow \infty \quad \Rightarrow \quad J \rightarrow 0
\end{align}
%
If the extrinsic forces are equal to zero, then the cost tends towards zero as the gear-ratio $R$ tends toward zero:
%
\begin{align}
\tau_E = 0 \quad \& \quad R \rightarrow 0 \quad \Rightarrow \quad J \rightarrow 0
\end{align}
%
If both the extrinsic forces and intrinsic forces are equal to zero, then the cost is zero for any gear-ratio:
%
\begin{align}
\tau_E = 0 \quad \& \quad \tau_I = 0 \quad \Rightarrow \quad J = 0  \quad \forall R
\end{align}
%


\subsection{Multiple DoF}
\label{sec:optgearproofn}

Starting with the EoM in inverse dynamic form (from eq. XXX):
%
\begin{align}
\vec{ \tau } &=  R^{-1} \vec{\tau}_E + R \vec{\tau}_I
\end{align}
%
Using the following quadratic cost function:
%
\begin{align}
J &=  \vec{ \tau }^T \vec{ \tau }
\end{align}
%
Finding the gear-ratios matrix $R$ that minimize this cost can be formulated as
%
\begin{align}
R^* &=  \operatornamewithlimits{argmin}\limits_{R} J \\
    &   \text{s.t.} \quad R_{i,j} \in \Re  \quad \& \quad R_{i,j} > 0  \quad \& \quad 
\end{align}
%

\paragraph{Derivation}

Using index notation, the EoM and cost function can be written as:
%
\begin{align}
\vec{ \tau } =  R^{-1} \vec{\tau}_E + R \vec{\tau}_I \quad &\Rightarrow \quad \tau_i = \sum_j{ \left[ R^{-1}\right]_{i,j} \tau_j^E + R_{i,j} \tau_j^I }\\
J =  \vec{ \tau }^T \vec{ \tau } \quad &\Rightarrow \quad J = \sum_i{ \tau_i^2 }
\end{align}
%
Note that here, superscript instead of subscript are used to identify extrinsic and intrinsic forces, to avoid confusion with indexes. Then the properties due to the diagonality of matrix $R$ can be used:
%
\begin{align}
R_{i,j}                    &= 0 \quad \forall \quad i \neq j \\
\left[ R^{-1}\right]_{i,j} &= 0 \quad \forall \quad i \neq j \\
\left[ R^{-1}\right]_{i,i} &= \left( R_{i,i}  \right)^{-1}
\end{align}
%
Then the equations can be simplified to:
%
\begin{align}
\tau_i &= \left( R_{i,i}  \right)^{-1} \tau_i^E + R_{i,i} \tau_i^I \\
J      &= \sum_i{ \left[  \left( R_{i,i}  \right)^{-1} \tau_i^E + R_{i,i} \tau_i^I   \right]^2 }
\end{align}
%
By inspection, it is possible to see that the cost $J$ is the sum of $n$ independent terms (one per DoF), and that given the assumptions those terms are independent. Hence, the cost $J$ can be minimized by minimizing individually each term with the appropriate $R_{i,i}$. The solution for minimizing each of those term is identical to the one for a single DoF robot, see section \ref{sec:optgearproof1}. Leading to 
%
\begin{align}
R_{i,i}^* = \sqrt{\left| \frac{\tau_i^E}{\tau_i^I}\right|}
\end{align}
%

\paragraph{Solution}
The optimal gear-ratio matrix, is thus constructed from independent solutions on each DoF:
%
\begin{align}
R^* = \left \{
\begin{array}[pos]{l}
	R^*_{i,j} = \sqrt{\left| \frac{\tau_i^E}{\tau_i^I}\right|} \quad \forall \quad i = j \\
	R^*_{i,j} = 0                                              \quad \forall \quad i \neq j
\end{array} \right.
\end{align}
%
Leading to the following total minimum cost:
%
\begin{align}
J^*   &= 2 \sum_i{ \left[ \tau_i^E \tau_i^I + \left| \tau_i^E \tau_i^I \right| \right] }
\end{align}
%

\newpage
\section{Stability Proofs}
\label{sec:stabproofs}

 
\subsection{R* Computed Torque Controller}
\label{sec:stabrstar}


\subsection{R* Sliding Mode Controller}
\label{sec:stabrstar}


\section{Chattering Bounds}
\label{sec:chat}

